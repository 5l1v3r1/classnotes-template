% !TeX document-id = {77664564-445f-48fb-8609-8a7b2df5e747}
% !TeX TXS-program:compile = txs:///pdflatex/[--shell-escape]

% ----------------------------------
% -------- Author and title --------
% ----------------------------------
\title{Class notes template}
\author{Borja Vega Mart\'in}
\date{\today}

% ----------------------------------
% ------------ Class ---------------
% ----------------------------------
\documentclass[9pt, a4paper]{../myTemplates/notesR3NT0N}

\usepackage{lipsum}

% ----------------------------------
% ----- Sections to include --------
% ----------------------------------
%\includeonly{install-miktex,conf-textstudio}

\begin{document}
\sloppy     % Evita que se corten las palabras y en su lugar justifica el texto

%-1 	\part{part}
% 0 	\chapter{chapter}
% 1 	\section{section}
% 2 	\subsection{subsection}
% 3 	\subsubsection{subsubsection}
% 4 	\paragraph{paragraph}
% 5  	\subparagraph{subparagraph}

%\maketitle             % Genera título
%\tableofcontents{}     % Genera índice de contenidos


%\input{chapter1}
%\input{section1}
\chapter{Note class template}
\section{Lorem Ipsum}
% Párrafos del 0 al 10 de Lorem Ipsum
\lipsum[0-8]

% ----------------------------------
% --------- Bibliography -----------
% ----------------------------------
\section{Bibliography}
\begin{itemize}
   \item Book 1, \textit{author1}.
   \item Book 2, \textit{author2}.
\end{itemize}

% Bibliografía usando BiBTeX
%---------------------------
% Los elementos que no se citan no aparecen en la lista de la bibliografía
%\cite{einstein} \cite{knuthwebsite} \cite{latexcompanion}
%\medskip
%\bibliographystyle{unsrt}  % Elegir estilo de bibliografía
%\bibliography{references}  % Nombre del fichero que contiene los elementos bibliográficos

\end{document}
